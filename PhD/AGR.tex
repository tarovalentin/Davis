\documentclass[a4paper,12pt]{article}
\usepackage{jheppub}
\usepackage{taro}
\usepackage{booktabs}

\title{Modern amplitude techniques}

\author[a]{Taro V. Brown}

\affiliation[a]{Department of Physics, UC Davis, One Shields Avenue, Davis, CA 95616, USA }


% e-mail addresses: one for each author, in the same order as the authors
\emailAdd{taro.brown@nbi.ku.dk}


\abstract{Notes on modern amplitude techniques written as part of a research project with Jaroslav Trnka.}

\begin{document} 
\maketitle
\flushbottom
\newpage
\section{Problem 1}
We will take the rotated null coordinates given by,
\begin{equation}
\begin{aligned}
u&=t-x~~~~~~~~~~&v=t+x\\
U&=\atan u~~~~~~~~~~&V=\atan v\\
T&=V+U,~~~~~&X=V-U
\end{aligned}
\end{equation}
Which combined give
\begin{equation}
T=\atan[t+x]+\atan[t-x],~~~~~~~~X=\atan[t+x]-\atan[t-x]
\end{equation}
\section{Problem 2}
We will take the rotated null coordinates given by,
\begin{equation}
	\begin{aligned}
		u&=t-x~~~~~~~~~~&v=t+x\\
		u'& u~~~~~~~~~~&V=\atan v\\
		T&=V+U,~~~~~&X=V-U
	\end{aligned}
\end{equation}
Which combined give
\begin{equation}
	T=\atan[t+x]+\atan[t-x],~~~~~~~~X=\atan[t+x]-\atan[t-x]
\end{equation}
%%%%%%%%%%%%%%%%%%%%%%%%%%%%%%%%%%%%%%%%%%%%%%%%%%%%%%%%%%%%%%%%%%%%%%%%

\end{document}

