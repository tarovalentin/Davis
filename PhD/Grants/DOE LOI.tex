\documentclass[]{article}

%opening
\title{}
\author{Taro V. Brown}

\begin{document}

\maketitle

\section*{Technical Abstract}
During the last two decades, significant strides have been made in the study of scattering amplitudes. New modern on-shell techniques have made calculations with a large number of particles and loops feasible. Many of the significant advancements in the field has come by studying $\mathcal{N} = 4$ super Yang-Mills (sYM) and using its planar, or large $N$, limit as a sandbox to explore the underlying geometric structure.
As examples of the discoveries made studying this theory, one can mention Yangian symmetry, dual conformal symmetry, integrability, a dual interpretation of amplitudes in terms of Wilson loops, as well as a plethora of other applications.

In recent years the geometric structure of scattering amplitudes in $\mathcal{N}=4$ sYM has been explored using the positive Grassmannian, on-shell diagrams, and the Amplituhedron. A natural progression is to try and extend these discoveries beyond planar $\mathcal{N}=4$ sYM to explore whether the geometric interpretation is a more general feature of quantum field theories (QFT's). Similarly, considering theories with $\mathcal{N}\neq4$, such as supergravity ($
\mathcal{N}=8$), will teach us to what extend these geometric pictures apply to other field theories. Our research will focus on these extensions and their applications, hoping that exploring the mathematical foundation will teach us about the structure of QFT's as a whole. We will use modern amplitude techniques such as the ones mentioned above, i.e. on-shell diagrams, the Grassmannian and the Amplituhedron.
\end{document}
